\section{Datenstruktur}
\subsection{Grunds"atzliches in aller K"urze}
\begin{itemize}
\item \textbf{Approximationsrate:} Um schneller an Ergebnisse zu kommen, werden Karten verkleinert und somit sind weniger Daten zu verarbeiten.
\item \textbf{Automat, Stadtteil, Koordinate:} Klassen, in welchen die wichtigsten Daten der einzelnen Automaten, Stadtteile, Koordinaten gespeichert sind.
\item \textbf{Karte:} Karte, die aus einzelnen Pixeln besteht, und f"ur jeden Pixel einen Wert enth"alt.
\item \textbf{Welt:} Speichert die wichtigsten Parameter, wie die einzelnen Stadtteile, die Pixelkarte, den Radius der Automaten oder die Anzahl der Automaten, jedoch keine L"osung des Problems. Die Positionen der einzelnen Automaten werden hier also \emph{nicht} gespeichert.
\item \textbf{Zustand:} Ausgehend von einer Welt wird hier gespeichert, wo die Automaten stehen und welches Gewicht sie haben.
\end{itemize}

\subsection{Automat}
Wie der Name schon sagt beschreibt die Klasse \texttt{in"-fo"-ma"-ti"-cup."-da"-ten"-struk"-tur."-Au"-to"-mat} einen Automaten. Es werden keine Berechnungen durchgef"uhrt, sondern nur Informationen "uber einen einzelnen Automaten gespeichert. Dazu geh"oren Position und Radius des Automaten, sowie ein Wahrheitswert, der angibt, ob der Automat gesperrt ist. Wenn ein Automat gesperrt ist, darf er nicht gel"oscht werden und seine Position darf nicht ver"andert werden.

\subsection{Koordinate}
Die Klasse \texttt{in"-for"-ma"-ti"-cup."-da"-ten"-struk"-tur."-Ko"-or"-di"-na"-te} wird f"ur die Speicherung der Positionen der Automaten sowie der Punkte der Polygone der Stadtteile verwendet. Die Klasse enth"alt keine Algorithmen oder Berechnungsschritte. 

\subsection{Pixelkarte}
\paragraph{L"osungsidee}
Die Pixelkarte ist eine Konvertierung der Welt als Vektorkarte in eine Pixeldarstellung. Sie enth"alt einzelne Stadtteile, deren Begrenzungen keine Vektoren mehr sind, sondern Striche auf einer Pixelkarte. 

Bei der Erzeugung der Pixelkarte wird die Karte automatisch skaliert (approximiert). Wird beispielsweise ein Approximationsfaktor von \(2\) angegeben, so werden s"amlichte vorkommende Koordinaten durch \(2\) geteilt. Die Aufl"osung der Karte wird also effektiv durch den Faktor \(4\) geteilt. Der Approximationsfaktor kann entweder manuell eingegeben werden oder automatisch ermittelt werden. Dadurch kann das Programm auch extreme Eingabedateien mit einer gro"sen Kartenbreite oder -h"ohe verarbeiten. Gro"se Eingabedateien werden einfach heruntergerechnet auf kleinere Eingabedateien. Nat"urlich nimmt dadurch auch die Aufl"osung und damit die Genauigkeit der L"osung ab. Die Geschwindigkeit der Algorithmen nimmt daf"ur zu. Wie genau der Zusammenhang zwischen der Geschwindigkeit der Algorithmen und der Gr"o"se der Karte aussieht, kann den Laufzeitabsch"atzungen der einzelnen Algorithmen entnommen werden.

Die Pixelkarte speichert f"ur die einzelnen Pixel keine Farben, sondern Gewichte. Ein Pixel kann sich in einem bestimmten Stadtteil befinden, dem nach der Gewichtungsfunktion der Attribute ein bestimmter Gewichtswert zugewiesen wurde. Jeder Pixel in einem Stadtteil erh"alt das Gewicht des betreffenden Stadtteils geteilt durch die Anzahl an Pixel im Stadtteil als Wert. Summiert man nun beispielsweise nur die H"alfte der Pixel des Stadtteils auf, erh"alt man nur die halbe Bewertung des Stadtteils.

\paragraph{Programm-Dokumentation}
Die Koordinaten der einzelnen Stadtteile wurden zuvor vom Parser erstellt, dabei wurden einfach die Werte aus der Eingabedatei "ubernommen. Die Methode \texttt{in"-for"-ma"-ti"-cup."-Welt"-funk"-ti"-on."-Approxi"-ma"-ti"-on} skaliert (approximiert) die Karte um einen bestimmten Approximationsfaktor. Dazu werden einfach alle Koordinaten der Polygone aller Stadtteile durch den Approximationsfaktor geteilt. Ebenso wird der Radius der Automaten durch den Approximationsfaktor geteilt. 

Nun beginnt die eigentliche Berechnung der Pixelkarte. Dazu werden Java-interne Funktionen verwendet, da diese einseits sehr schnell und andererseits bei extremen Eingabedaten (z.B. Stadtteilen, die aus keinem Pixel bestehen) nicht anf"allig f"ur Fehler sind\footnote{Diesen Teil hatten wir urspr"unglich komplett selbst geschrieben.}. Die Funktion \texttt{in"-for"-ma"-ti"-cup."-Welt"-funk"-ti"-on."-er"-stelle"-Pix"-el"-kar"-te} erstellt die eigentliche Pixelkarte. Dazu werden die Stadtteile als gef"ullte Polygone auf ein \texttt{Buffered"-Image} geschrieben. Danach werden die einzelnen Pixel aus dem Bild ausgelesen und in ein Array geschrieben, auf das effizienter zugegriffen werden kann. Anf"angliche Probleme gab es, weil das Gewicht durch die Anzahl der Pixel geteilt wird und dadurch Werte enstanden, die kleiner als 1 sind. Das \texttt{Buffered"-Image} unterst"utzt jedoch nur Ganzzahlen. Als L"osung wird ein zweites \texttt{Buffered"-Image} erstellt, das angibt, durch welchen Wert jeder Pixel geteilt werden soll.

\paragraph{Laufzeitabsch"atzung}
Das Skalieren (Approximieren) der Karte h"angt lediglich linear von der Anzahl der Koordinaten ab, die verarbeitet werden sollen. Wie schnell das Zeichnen der Polygone geht, ist uns nicht genau bekannt, da diese Aufgabe von Java-Bibliotheken "ubernommen wird. Das "Ubertragen der einzelnen Pixel-Werte in das Array h"angt linear von der Anzahl der Pixel und damit der Gr"o"se der Karte ab. Die Skalierung beschleunigt diesen Prozess nur um einen konstanten Faktor. 
\subsection{Stadtteil}
Die Klasse \texttt{in"-for"-ma"-ti"-cup."-da"-ten"-struk"-tur."-Stadt"-teil} wird f"ur die Speicherung der Stadtteile verwendet. F"ur jeden Stadtteil werden die Name und Werte der einzelnen Attribute gespeichert, sowie die Eckpunkt der Polygone, die den Stadtteil abgrenzen, wobei diese Punkte Koordinaten auf der Pixelkarte sind. Das Gewicht eines Stadtteils, das aus der Gewichtung der Attribute ergibt, wird in der Variable \texttt{\_ge"-wicht"-Nach"-Fkt} gespeichert.

Der einzige Berechnungsschritt in dieser Klasse ist die Berechnung der Fl"ache des Polygons (Methode \texttt{be"-rech"-ne"-Ne"-ben"-effek"-te}). Diese ergibt sich nach der gau"sschen Trapezformel.
\begin{equation}2 A=\sum_{i=1}^n (y_i + y_{i+1})(x_i-x_{i+1})\end{equation}
Die Laufzeit dieses Berechnungsschrittes h"angt linear von der Anzahl der Punkte eines Polygons ab.


\subsection{Welt}
Die Klasse \texttt{in"-for"-ma"-ti"-cup."-da"-ten"-struk"-tur."-Welt} beschreibt die Probleminstanz ohne den Zustand, also ohne die L"osung (Automaten). Die Welt besteht aus den einzelnen Stadtteilen, dem Radius der Automaten, der Anzahl der zu platzierenden Automaten und der Pixelkarte. Die L"osung des Problem wird sp"ater in einen \texttt{Zu"-stand} geschrieben. In der \texttt{Welt} wird au"serdem der Approximationsfaktor gespeichert. Die \texttt{Welt} wird jedoch nur approximiert, wenn die entsprechende Methode in der \texttt{Welt"-funk"-ti"-on} aufgerufen wird.

\subsection{Gewichtskarte}

\paragraph{L"osungsidee}
Da viele Algorithmen simulieren, wie Automaten an den verschiedensten Stellen bewertet w"urden, war eine "Uberlegung um die Programmeffizienz zu steigern, eine zweidimensionale Karte vorauszuberechnen, die f"ur jede Koordinate die erwartete Bewertung beinhaltet. Da der Radius eines Automaten durch die Eingabedatei festgelegt ist, kann diese von fast jedem Algorithmus ben"otigte Gewichtskarte vorausberechnet werden.
Urspr"unglich wurde f"ur jede Koordinate \texttt{(x, y)} das umliegende Quadrat mit der Kantenl"ange des Kreisdurchmessers \texttt{[(x-r..x+r),(y-r..y+r)]} betrachtet: Mittels des Satzes des Pythagoras wurde bestimmt, ob der Punkt innerhalb des umliegenden Kreises liegt, und falls dies zutrifft, das Gewicht des darunter befindlichen Stadtteiles \texttt{getGewicht(x, y)} zu einer spezifischen Summe der Koordinate hinzuaddiert.
Dieses Verfahren war "au"serst ineffizient, da f"ur jede m"ogliche Koordinate viele Berechnungen durchgef"uhrt werden mussten.
Um dies zu vermeiden, wurde die folgende Optimierung vorgenommen: In einer separaten Variable der Gr"o"se \texttt{short Kreis = [2r, 2r]} wurde ein derartiger Kreis mittels pythagor"aischer Betrachtung vorberechnet. Sofern der Punkt im Kreis lag (also die Bedingung $(x-r)^2+(y-r)^2 \leq r^2$ erf"ullt ist) enth"alt das Kreisarray an der Stelle \texttt{(x,y)} den Wert \texttt{1}, ansonsten \texttt{0}. Aufsummiert wurde nun nicht mehr \texttt{getGewicht(x, y)}, sondern \texttt{getGewicht(x, y) * Kreis[x][y]}.
Um die ben"otigte Zeit zur Berechnung der Gewichtskarte weiter zu reduzieren, wurden die so genannten \textbf{Deltakreise} eingef"uhrt. Es ist n"amlich zu erkennen, dass das Gewicht des Pixels \texttt{(x+1, y)} auch aus dem Gewicht des Punktes \texttt{(x, y)} berechnet werden kann. Selbiges gilt f"ur \texttt{(x, y+1)} und \texttt{(x, y)}. Deshalb werden vergleichbare Arrays (\texttt{short deltakreis\_rechts = [2r+1, 2r]} und \texttt{short deltakreis\_unten = [2r, 2r+1]}) vorausberechnet, um die Anzahl der notwendigen Operationen weiter zu reduzieren. Dem Bildbereich dieser beiden Arrays wurde der Wert \texttt{-1} hinzugef"ugt. Als Berechnungsanweisung wird der gleiche Algorithmus wie f"ur den normalen Kreis angewandt (\textit{Multiplikation des Gewichts mit dem Inhalt des Arrays}).

% \paragraph{Programm-Dokumentation}

\paragraph{Laufzeitabsch"atzung}
Das Generieren der Gewichtskarte h�ngt von mehreren Faktoren ab: Von der Kartengr"o"se \texttt{Welt"-funk"-ti"-on."-get"-Max"-i"-ma"-le"-Brei"-te"-Hoehe(true) = A}, den Automatenradien \texttt{Welt.getRadiusAutomaten() = r} und dem Approximationsfaktor \texttt{Welt.getApproxrate() = a}. F"ur jeden Punkt der approximierten Karte, also insgesamt $ \frac{A}{a} $ Punkte, m"ussen die Referenz-/Deltakreis-Arrays durchlaufen werden, welche widerum eine Gr"o"se von $ 4 * r^2 $, bzw. vernachl"assigbar $ 4 * r * (r+1) $ besitzen. Daraus folgt, dass je Koordinate $ \frac{A * 4 * r^2}{a} $ Vergleichsoperationen durchgef"uhrt werden m"ussen. Das Ergebnis der Vergleichsoperation bestimmt, ob ein Wert addiert (subtrahiert) werden muss. Durchschnittlich findet dies alle $ \frac{2*2}{\pi} = \frac{4}{\pi} $ Koordinaten statt.

\subsection{Zustand}
Die Klasse \texttt{in"-for"-ma"-ti"-cup."-da"-ten"-struk"-tur."-Zu"-stand} beschreibt die L"osung einer Probleminstanz, also wo die einzelnen Automaten platziert worden sind. Dazu gibt es ein Array \texttt{\_au"-to"-ma"-ten}. Im Zustand werden au"serdem die Bewertung der aktuellen L"osung und ein Zeiger auf die Gewichtskarte gespeichert. Die Gewichtskarte ist nur "uber den Zustand erreichbar und wird beim ersten Zugriff erstellt und gespeichert. Erfolgen sp"ater weitere Zugriffe auf die Gewichtskarte, muss diese nicht neu berechnet werden. In dieser Klasse befinden sich deshalb auch der Algorithmus zur Berechnung der Gewichtskarte. 

Vor der Berechnung der Automatenpositionen durch die Algorithmen wird bereits ein Automaten-Array mit der korrekten Gr"o"se erstellt, welches jedoch nur \texttt{null}-Zeiger enth"alt. Somit kann bereits vor Ausf"uhrung eines Algorithmus durch die GUI auf das Automaten-Array zugegriffen werden, um einzelne Automaten per Hand zu erstellen. 

