\subsection{Welt}
Die Klasse \texttt{in"-for"-ma"-ti"-cup."-da"-ten"-struk"-tur."-Welt} beschreibt die Probleminstanz ohne den Zustand, also ohne die L"osung (Automaten). Die Welt besteht aus den einzelnen Stadtteilen, dem Radius der Automaten, der Anzahl der zu platzierenden Automaten und der Pixelkarte. Die L"osung des Problem wird sp"ater in einen \texttt{Zu"-stand} geschrieben. In der \texttt{Welt} wird au"serdem der Approximationsfaktor gespeichert. Die \texttt{Welt} wird jedoch nur approximiert, wenn die entsprechende Methode in der \texttt{Welt"-funk"-ti"-on} aufgerufen wird.
