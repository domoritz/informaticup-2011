\subsection{Hauptfenster}
Das Hauptfenster \texttt{in"-for"-ma"-ti"-cup."-In"-for"-ma"-ti"-cup"-View} ist die zentrale Steuerkomponente des Programms. Von dort aus k"onnen "uber das Men"u neue Berechnungen gestartet werden und die L"osung kann in eine Textdatei exportiert werden. 

Nach der Berechnung wird der Zustand der Welt mit allen Automaten und Stadtteilen auf ein \texttt{JPanel} gezeichnet. Dabei handelt es sich jedoch nicht um regul"ares Panel, sondern um eine Erweiterung zum \texttt{in"-for"-ma"-ti"-cup."-drawing"-Panel}. Dieses modifizierte Panel enth"alt Methoden zum Zeichnen von Stadtteilen, Automaten und der Gewichtskarte. Ebenso reagiert das Panel auf Mauseingaben. Mit der Maus k"onnen neue Automaten erzeugt und bestehende Automaten verschoben, gel"oscht und gesperrt werden. Bei diesen Aktionen ist es wichtig, die Mauskoordinaten in Pixelkoordinaten der Pixelkarte umzurechnen. Das Panel erm"oglicht au"serdem das Scrollen. Wenn das Fenster zu klein f"ur die Karte ist, wenn der Slider zur Ausgabeskalierung nach rechts geschoben wurde, kann an jede beliebige Position der Karte scrollen. Eine weitere wichtige Eigenschaft des Panels ist das automatische Neu-Zeichnen des Inhaltes, zum Beispiel wenn die Gr"o"se des Fensters ver"andert wurde.

"Uber die Men"uleiste des Hauptfensters kann der Benutzer au"serdem verschiedene Modi der L"osungsbearbeitung aktivieren und deaktivieren (Automaten erstellen, l"oschen und sperren). 
