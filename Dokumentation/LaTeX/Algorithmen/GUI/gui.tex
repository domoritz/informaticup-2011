\section{"Ubersicht "uber die Programmoberfl"ache}
Dieses Kapitel beschreibt, wie die GUI in Java technisch umgesetzt wurde und ist bewusst kurz gehalten, da sie f"ur die meisten \emph{Leser} wohl eher von uninteressant ist. Beim Entwurf der Oberfl"ache versuchten wir, die folgenden Grunds"atze so gut wie m"oglich zu beachten.
\begin{itemize}
\item \textbf{Zweisprachigkeit:} Das Programm enth"alt eine englischsprachige sowie eine deutschsprachige "Ubersetzung der meisten GUI-Texte. Dies wurde "uber Ressourcen-Dateien (\texttt{in"-for"-ma"-ti"-cup."-res"-ources.*}) gel"ost. 
\item \textbf{Falsche Bedienung vermeiden:} So weit es m"oglich ist, werden falsche Eingaben durch den Benutzer erst gar nicht erlaubt. So kann der Benutzer beispielsweise die Berechnung erst gar nicht starten, wenn er einen ung"ultigen Wert f"ur die Temperatur eingibt. Au"serdem kann er Automaten nicht so platzieren, dass eine ung"ultiger Zustand entsteht.
\item \textbf{Trennung von GUI und Algorithmus:} Durch die Aufteilung in Packages und verschiedene Klassen ist uns das relativ gut gelungen. Lediglich die Klasse \texttt{in"-for"-ma"-ti"-cup."-In"-for"-ma"-ti"-cup"-View} enth"alt die Steuerung des Programmablaufes. Dort werden zum Beispiel Parameter f"ur die Algorithmen ausgewertet und die einzelnen Berechnungsschritte (z.B. Algorithmen) angest"osen. Dadurch konnten wir uns jedoch Arbeit sparen, weil oft auf Parameter-Werte aus der GUI zur"uckgegriffen werden muss.
\end{itemize}

\subsection{Hauptfenster}
Das Hauptfenster \texttt{in"-for"-ma"-ti"-cup."-In"-for"-ma"-ti"-cup"-View} ist die zentrale Steuerkomponente des Programms. Von dort aus k"onnen "uber das Men"u neue Berechnungen gestartet werden und die L"osung kann in eine Textdatei exportiert werden. 

Nach der Berechnung wird der Zustand der Welt mit allen Automaten und Stadtteilen auf ein \texttt{JPanel} gezeichnet. Dabei handelt es sich jedoch nicht um regul"ares Panel, sondern um eine Erweiterung zum \texttt{in"-for"-ma"-ti"-cup."-drawing"-Panel}. Dieses modifizierte Panel enth"alt Methoden zum Zeichnen von Stadtteilen, Automaten und der Gewichtskarte. Ebenso reagiert das Panel auf Mauseingaben. Mit der Maus k"onnen neue Automaten erzeugt und bestehende Automaten verschoben, gel"oscht und gesperrt werden. Bei diesen Aktionen ist es wichtig, die Mauskoordinaten in Pixelkoordinaten der Pixelkarte umzurechnen. Das Panel erm"oglicht au"serdem das Scrollen. Wenn das Fenster zu klein f"ur die Karte ist, wenn der Slider zur Ausgabeskalierung nach rechts geschoben wurde, kann an jede beliebige Position der Karte scrollen. Eine weitere wichtige Eigenschaft des Panels ist das automatische Neu-Zeichnen des Inhaltes, zum Beispiel wenn die Gr"o"se des Fensters ver"andert wurde.

"Uber die Men"uleiste des Hauptfensters kann der Benutzer au"serdem verschiedene Modi der L"osungsbearbeitung aktivieren und deaktivieren (Automaten erstellen, l"oschen und sperren). 

\subsection{Nebenfenster}
Neben dem Hauptfenster gibt es weitere Fenster, wie die About-Box mit den Namen der Autoren und Dialog mit den Parameter-Einstellungen f"ur die Berechnung (Berechnungsdialog). Der Berechnungsdialog \texttt{in"-for"-ma"-ti"-cup."-create"-Calc"-Frame} enth"alt Textfelder, Optionsbuttons und Checkboxen f"ur die Auswahl des Algorithmus und der Festlegung der Berechnungsparameter. Eine Tabelle \texttt{JTable} enth"alt die Gewichtungen der einzelnen Attribute, die der Benutzer frei w"ahlen kann. 

