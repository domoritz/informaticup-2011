\section{Typischer Programmablauf}
Ein typischer Programmablauf mit Anwendung der Algorithmen und Generieren einer Ausgabedatei sieht wie folgt aus.
\begin{enumerate}
\item Benutzer l"adt eine Eingabedatei durch Ausw"ahlen einer Datei im Dateiauswahl-Dialog.
\begin{enumerate}
\item Parser liest die Eingabedatei ein und erzeugt eine \texttt{Welt} (\texttt{in"-for"-ma"-ti"-cup."-create"-Calc"-Frame."-start"-Parsing}).
\item Attribute werden aus der \texttt{Welt} ausgelesen und in die Tabelle hinzugef"ugt (\texttt{in"-for"-ma"-ti"-cup."-create"-Calc"-Frame."-load"-File})
\end{enumerate}
\item Benutzer gewichtet die Attribute in der Tabelle (optional).
\item Benutzer startet den Berechnungvorgang (\texttt{in"-for"-ma"-ti"-cup."-In"-for"-ma"-ti"-cup"-View."-start"-Cal"-cu"-lation}).
\begin{enumerate}
\item Gewichtung der einzelnen Attribute wird ausgewertet und die Stadtteile erhalten ihr Gewicht \texttt{\_ge"-wicht"-Nach"-Fkt} (\texttt{in"-for"-ma"-ti"-cup."-create"-Calc"-Frame."-be"-rei"-te"-Be"-rech"-nung"-Vor}).
\item Parameter und ausgew"ahlte Algorithmen werden ausgewertet, Variablen werden gesetzt (\texttt{in"-for"-ma"-ti"-cup."-In"-for"-ma"-ti"-cup"-View."-start"-Cal"-cu"-lation}), neuer Task f"ur die Berechnung wird erstellt (\texttt{in"-for"-ma"-ti"-cup."-In"-for"-ma"-ti"-cup"-View."-Start"-Cal"-cu"-lation"-Task."-do"-In"-Back"-ground}).
\item Koordinaten werden approximiert/skaliert (\texttt{in"-for"-ma"-ti"-cup."-Welt"-funk"-ti"-on."-Approx"-i"-ma"-ti"-on}).
\item Wenn kein Algorithmus ausgew"ahlt wurde und die Karte nur gezeichnet werden soll: Neuer \texttt{Zu"-stand} wird erstellt, das hei"st, bereits gesetzte Automaten werden verworfen.
\item Pixelkarte wird erstellt (\texttt{in"-for"-ma"-ti"-cup."-Welt"-funk"-ti"-on."-er"-stelle"-Pix"-el"-kar"-te}).
\item Wenn bisher noch kein \texttt{Zu"-stand} existiert: Erstelle neuen \texttt{Zu"-stand}. Bestehende Automaten werden nicht gel"oscht.
\item Gewichtskarte wird erzeugt (\texttt{in"-for"-ma"-ti"-cup."-da"-ten"-struk"-tur."-Zu"-stand."-er"-zeu"-ge"-Ge"-wichts"-kar"-te}).
\item F"uhre Er"offnungsalgorithmus aus (Greedy, Greedy mit Strichproben, Zufall oder Backtracking).
\item F"uhre Optimierungsalgorithmus aus (Simulierte Abk"uhlung, Tabu-Suche, und/oder Einzelverschiebung).
\item Zeichne die L"osung (Klasse \texttt{in"-for"-ma"-ti"-cup."-draw"-ing"-Panel}).
\end{enumerate}
\item Benutzer speichert die L"osung (\texttt{in"-for"-ma"-ti"-cup."-Da"-tei"-ex"-port."-da"-tei"-Aus"-ge"-ben}).
\end{enumerate}

\subsection{Dateieingabe und Dateiausgabe}
Eine Datei wird sofort geladen, wenn der Benutzer die Datei im Dateiauswahl-Dialog "offnet. Die Klasse \texttt{in"-for"-ma"-ti"-cup."-Da"-tei"-im"-port} ist daf"ur zust"andig, die komplette Datei erst einmal in eine Zeichenfolge zu lesen. Diese kann dann vom Parser \texttt{in"-for"-ma"-ti"-cup."-Par"-ser} zu einer \texttt{Welt} verarbeitet werden. Dabei werden die einzelnen Stadtteile erzeugt, die Attribute ausgelesen und Radius und Anzahl der Automaten ermittelt. Schl"agt dieser Vorgang fehl, wird eine \texttt{Par"-ser"-Ex"-ception} geworfen.

Wenn alle Berechnung abgeschlossen sind, liegt ein \texttt{Zu"-stand} vor, der vom Benutzer auch nachtr"aglich noch ver"andert werden kann. Er kann beispielsweise Automaten verschieben. Ein solcher \texttt{Zu"-stand} kann durch die Methoden in der Klasse \texttt{in"-for"-ma"-ti"-cup."-Da"-tei"-ex"-port} dann in eine Ausgabedatei geschrieben werden, die die in der Aufgabenstellung geforderte Form hat. 

\subsection{Weltfunktion}
Die Klasse \texttt{in"-for"-ma"-ti"-cup."-Welt"-funk"-ti"-on} enth"alt Operationen und Funktionen, die auf die \texttt{Welt} angewandt werden k"onnen. Dabei handelt es sich um das Erstellen der Pixelkarte (siehe Kapitel \emph{Zustand}) und das Approximieren der Koordinaten. 

Au"serdem existiert eine Funktion, die die maximale Breite bzw. H"ohe der Karte ermittelt. Dieser Wert ist notwendig, um automatisch einen guten Wert f"ur die Approximation zu finden. Die Karte wird dann so approximiert, dass bei jeder Karte etwa gleich gro"se Pixelkarten entstehen, die in vern"unftiger Zeit verarbeitet werden k"onnen. Gro"se Karten werden also st"arker verkleinert als kleine Karten. 

