\subsection{Dateieingabe und Dateiausgabe}
Eine Datei wird sofort geladen, wenn der Benutzer die Datei im Dateiauswahl-Dialog "offnet. Die Klasse \texttt{in"-for"-ma"-ti"-cup."-Da"-tei"-im"-port} ist daf"ur zust"andig, die komplette Datei erst einmal in eine Zeichenfolge zu lesen. Diese kann dann vom Parser \texttt{in"-for"-ma"-ti"-cup."-Par"-ser} zu einer \texttt{Welt} verarbeitet werden. Dabei werden die einzelnen Stadtteile erzeugt, die Attribute ausgelesen und Radius und Anzahl der Automaten ermittelt. Schl"agt dieser Vorgang fehl, wird eine \texttt{Par"-ser"-Ex"-ception} geworfen.

Wenn alle Berechnung abgeschlossen sind, liegt ein \texttt{Zu"-stand} vor, der vom Benutzer auch nachtr"aglich noch ver"andert werden kann. Er kann beispielsweise Automaten verschieben. Ein solcher \texttt{Zu"-stand} kann durch die Methoden in der Klasse \texttt{in"-for"-ma"-ti"-cup."-Da"-tei"-ex"-port} dann in eine Ausgabedatei geschrieben werden, die die in der Aufgabenstellung geforderte Form hat. 
