\section{Installationsanleitung}
Wie in der Dokumentation beschrieben, setzten wir die Aufgabe in Java mithilfe der NetBeans
IDE um. Mit NetBeans ist es sehr einfach m"oglich, den Code zu untersuchen und das Programm neu zu kompillieren. 

Um das Programm zu starten, ist  aber nur die Java Runtime Environment (JRE) oder eine vergleichbare Java VM erforderlich. Um das Programm zu starten, gen"ugt es, die Datei \texttt{informaticup.jar} im Ordner \texttt{Programm} auszuf"uhren. Falls das Programm so nicht gestartet werden kann (wegen einer falsch gesetzten \texttt{PATH} Variable), l"asst sich das Programm auch mit dem Befehl \texttt{java -jar ``informaticup.jar``} starten. Wichtig ist, dass der Ordner \texttt{lib} mit den Bibliotheken immer neben der Programmdatei liegt, damit die Bibliotheken geladen werden k"onnen.